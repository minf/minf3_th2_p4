\documentclass{scrreprt}

\usepackage{amsmath, amsthm, amssymb}
\usepackage[utf8]{inputenc}
\usepackage{zed-csp}
\usepackage{enumerate}

\begin{document}

\author{Andreas Krohn, Benjamin Jochheim, Theodor Nolte, Benjamin Vetter}
\title{TH2 - Übung 4}

\maketitle

\chapter{CSP Laws}

\section{SCP Laws beweisen}

\begin{enumerate}[a)]

% 1.1 a)
\item{
  $ a \then (P_1 \intchoice P_2) =_T (a \then P_1) \intchoice (a \then P_2) $

  Gleichheit von Prozessen im Trace-Modell:\\
  $a \then (P_1 \intchoice P_2) =_T (a \then P_1) \intchoice (a \then P_2) \\
  \equiv traces(a \then (P_1 \intchoice P_2)) = traces((a \then P_1) \intchoice (a \then P_2))$ \\

  $\rightarrow$ zu zeigen: $traces(a \then (P_1 \intchoice P_2)) = traces((a \then P_1) \intchoice (a \then P_2))$

  \begin{align*}
    traces(a \then (P_1 \intchoice P_2)) &= \{ \langle \rangle \} \cup \{ \langle a \rangle \cat tr | tr \in traces(P_1 \intchoice P_2)\} & \text{Prefixing, 6.10}\\
                                         &= \{ \langle \rangle \} \cup \{ \langle a \rangle \cat tr | tr \in traces(P_1) \cup traces(P_2)\} & \text{Choice, 6.16}\\
                                         &= \{ \langle \rangle \} \cup \{ \langle a \rangle \cat tr | tr \in traces(P_1)\} &\text{Vereinigung von Mengen}\\
                                         &\hspace{0.5cm}\cup \{ \langle a \rangle \cat tr | tr \in traces(P_2)\}\\
                                         &= \{ \langle \rangle \} \cup \{ \langle a \rangle \cat tr | tr \in traces(P_1)\} &\text{Idempotenz von Mengen}\\
                                         &\hspace{0.5cm}\cup \{ \langle \rangle \} \cup \{ \langle a \rangle \cat tr | tr \in traces(P_2)\}\\
                                         &= traces(a \then P_1) \cup traces(a \then P_2) &\text{Prefixing, 6.10}\\
                                         &= traces((a \then P_1) \intchoice (a \then P_2)) &\text{(Internal-)Choice, 6.16}\\
  \end{align*}
  \flushright{\qedsymbol}\\
}

% 1.1 b)
\item{
  $ Skip \interrupt P =_T Skip \intchoice P $

  Gleichheit von Prozessen im Trace-Modell:\\
  $Skip \interrupt P =_T Skip \intchoice P \hspace{10pt} \equiv \hspace{10pt} traces(Skip \interrupt P) = traces(Skip \intchoice P) $ \\

  $\rightarrow$ zu zeigen: $traces(Skip \interrupt P) = traces(Skip \intchoice P)$

  \begin{align*}
    traces(Skip \interrupt P) &= traces(Skip) \cup \{tr_1 \cat tr_2 | tr_1 \in traces(Skip) \wedge \tick \notin \sigma(tr_1) & \text{Interrupt, 6.53}\\
                              &\hspace{4.9cm} \wedge tr_2 \in traces(P) \}\\
                              &= \{ \langle \rangle, \langle \tick \rangle \} \cup \{ tr_1 \cat tr_2 | tr_1 \in \{ \langle \rangle, \langle \tick \rangle \} \wedge \tick \notin \sigma(tr_1) & \text{Skip, 6.4}\\
                              &\hspace{4.9cm} \wedge tr_2 \in traces(P) \}\\
                              &|NB_1: tr_1 \in \{ \langle \rangle, \langle \tick \rangle \} \wedge \tick \notin \sigma(tr_1) \hspace{10pt} \Leftrightarrow \hspace{10pt} tr_1 = \{ \langle \rangle \}\\
                              &|NB_2: \{a \cat b\} = \{ b \} \Leftrightarrow a = \{ \langle \rangle \}\\
                              &= \{ \langle \rangle, \langle \tick \rangle \} \cup \{ \langle \rangle \cat tr_2 | tr_2 \in traces(P) \} & NB_1, NB_2\\
                              &= \{ \langle \rangle, \langle \tick \rangle \} \cup traces(P) & NB_2\\
                              &= traces(Skip) \cup traces(P) & \text{Skip, 6.4}\\
                              &= traces(Skip \intchoice P) & \text{Choice, 6.14}\\
  \end{align*}

  \flushright{\qedsymbol}\\
}
\end{enumerate}

\section{CSP Laws anwenden}

$
P1 = ((a \then c \then Skip) \extchoice (b \then c \then Skip))\\
P2 = (d \then Skip)\\
P3 = c \then d \then Skip
$

\begin{align*}
  \text{zu zeigen: } (P1; P2) \setminus \{a,b\} =_T P3
\end{align*}

\begin{align*}
  (P1; P2) \setminus \{a,b\} &= (((a \then c \then Skip) \extchoice (b \then c \then Skip)); (d \then Skip)) \setminus \{a,b\}& \\
         &=_T ((a \then c \then Skip) \intchoice (b \then c \then Skip); (d \then Skip)) \setminus \{a,b\} &\text{choice-equiv$_T$, 8.15}\\
         &= (a \then c \then Skip) \setminus \{a,b\} \intchoice (b \then c \then Skip) \setminus \{a,b\}; &\intchoice\text{-hide-dist$_T$, 8.81, NB}\\
         &\hspace{0.5cm}(d \then Skip) \setminus \{a,b\}\\
         &=(c \then Skip) \setminus \{a,b\} \intchoice (c \then Skip) \setminus \{a,b\}; &\text{3 x hide-step$_1$, 8.79}\\
         &\hspace{0.5cm} d \then Skip \setminus \{a,b\}\\
         &=c \then Skip \setminus \{a,b\} \intchoice c \then Skip \setminus \{a,b\}; &\text{2 x hide-step$_1$, 8.79}\\
         &\hspace{0.5cm} d \then Skip \setminus \{a,b\}\\
         &= (c \then Skip) \intchoice (c \then Skip); d \then Skip &\text{3 x hide-term, 8.85}\\
         &= c \then Skip; d \then Skip & \extchoice\text{-idem, 8.16}\\
         &= c \then d \then Skip & \text{;-unit-l, 8.94}\\
         &= P3
\end{align*}

\flushright{\qedsymbol}\\

\chapter{Failures Semantik}

\section{Stable Failures in Transitionsgraphen}

\end{document}
